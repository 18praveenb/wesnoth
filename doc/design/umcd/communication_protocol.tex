\chapter{Communication protocol}
\label{chapter:communication_protocol}

\section{Document Convention}

The following tokens will be used to further describe an item.

\begin{description}
\item[?]
  The current item is optional (request) or should not appear every time under specified conditions (reply).
\item[*]
  The current item appears 0, 1 or multiple times.
\item[+]
  The current item appears 1 or multiple times.
\end{description}
The description of an item can take different formats:
\begin{description}
  \item[\textless{}value\textgreater{}]
    A text between ``\textless{}\textgreater{}'' is replaced with the appropriate value in the current context.
  \item[\{value\}]
    A text between  ``\{\}'' is optional.
\end{description}


\section{Server configuration}
\label{section:communication_protocol:server}

On the server there are several configuration files:
\begin{itemize}
\item
	The main configuration file, umcd.cfg, this file is stored in the
	data directory, which is a parameter for the program. This file is
	further discussed in \cref{section:communication_protocol:server:umcd.cfg}.

\item
	In this data directory there is a subdirectory per addon. These
	directories have the following files:
	\begin{description}
	\item[umc.cfg]
		The config file for the UMC, this file is based on the client's
		pbl file. This file is further discussed in
		\cref{section:communication_protocol:server:umc.cfg}.

	\item[umc.gz]
		The gzipped UMC.

	\item[umc-\emph{lang}.cfg]
		The configuration file for the translation of the UMC. The
		\emph{lang} is replace with the locale of the translation.
		This file is further discussed in
		\cref{section:communication_protocol:server:umc-lang.cfg}.

	\item[umc-\emph{lang}.gz]
		The gzipped mo-file for the translation.

	\end{description}

\end{itemize}

\subsection{Compressed files}

If the compression ratio is bad, the file could not be zipped. But, to facilitate
sending data over the network, the files sent will be gather into a tarball (.tar).

\subsection{umcd.cfg}
\label{section:communication_protocol:server:umcd.cfg}

This file contains the settings of the UMCD.

\begin{lstlisting}
port = PORT
threads = THREADS
dir = DIR
\end{lstlisting}

The keys have the following meaning:
\begin{description}
\item[PORT]
	The TCP/IP port to listen to for incoming requests.

\item[THREADS]
	The number of working threads to start. A value of $0$ means start as
	many threads as there are detected hardware cores or hyperthreading units.

\item[DIR]
  The directory where the add-ons will be stored.
\end{description}

The values are not encapsulated into a global markup because we parse the options with \textit{Boost.Program\_options}.

\subsection{umc.cfg}
\label{section:communication_protocol:server:umc.cfg}

This file contains the configuration of the umc on the server. The
combination of this file together with the various
\mbox{umc-\emph{lang}.cfg} files form the master information regarding an
umc. All other files that specify the umc or its wire-protocol get their
information from these files. Therefore other sections will refer to this
section.

The file is specified like:
\begin{lstlisting}
[umc]
  [info]
    id = ID
    name = NAME
    description = DESCRIPTION
    type = TYPE
    version = VERSION
    authors = AUTHORS
    email = EMAIL
    password = PASSWORD
  [/info]
  [language]
    native_language = NATIVE_LANGUAGE
    translatable = TRANSLATABLE
  [/language]
	[dependencies]?
		[dependency]+
			id = ID
			version? = VERSION_MASK
		[/dependency]
	[/dependencies]
  [state]
    uploader_ip_address = UPLOADER_IP_ADDRESS
    timestamp = TIMESTAMP
    downloads = DOWNLOADS
    uploads = UPLOADS
  [/state]
[/umc]
\end{lstlisting}

The keys have the following meaning:
\begin{description}

\item [[info]] The main informations provided by the UMC-creator.

\begin{description}
\item[ID]
	The unique id of the content. When uploading this value determines
	whether an UMC is the same UMC.

\item[NAME]
  The name of the content. This string is not unique. This allows two
  versions of the UMC on the server. This can be used to allow a new beta
  next to a stable, as long as the ID differs.

\item[DESCRIPTION]
  The description of the UMC.

\item[TYPE]
	The type of the addon. At the moment the following values are allowed:
	\begin{description}
	\item[CAMPAIGN\_SP] A single player campaign.
	\item[CAMPAIGN\_MP] A multi-player player campaign.
	\item[ERA] A multi-player era.
	\item[MAP] A multi-player map.
	\item[MAP\_PACK] A multi-player map pack.
	\item[RPG\_SP]
		A RPG like single player game. (This also includes builder games in
		the style of settlers.)
	\item[RPG\_MP]
		A RPG like multi-player game. (This also includes builder games in
		the style of settlers.)
	\item[MEDIA]
		 Miscellaneous resources for UMC authors/users, for example, music packs, packages of general-purpose WML, etc.
	\item[OTHER] Everything that doesn't fit in one of the above.
	\end{description}

\item[VERSION]
	A string determining the version. The recommended string format like:
	`x.y.z' or `x.y.z-q'. The version is parsed as version number and 
comparisons are done considering all the number and letter
$a < z < A < Z < 0 < 9 < Empty$, all other characters are parsed as separator.
For example, ``aZ=4'' is greater than ``aZ'', and ``='' is the separator character.

\item[AUTHORS]
	A colon separated list of authors of the umc. When there is only one
	author no colon is required.

\item[EMAIL]
	The email address of the primary author or a mailinglist. When there are
	problems with the umc the Wesnoth team can use this address to contact
	the authors, so make sure it's valid and doesn't need registration.
	The address is not used in other ways nor shared with third-parties.

\item[PASSWORD]
	The password required to upload a new version of an UMC or remove an UMC
	from the server. See also \cref{section:security:passwordstorage} and \cref{section:security:passwordtransmission}.

\end{description}

\item [[language]] The items related to the language files of the UMC.

\begin{description}

\item[NATIVE\_LANGUAGE]
	The native language id of the UMC, and the language in which are the name and the description.

\item[TRANSLATABLE]
	This flag causes the UMC to be synchronized with Wescamp. It's advisable
	to only set the flags when the strings are somewhat stable. This flag
	can only be set if American English is available.
\end{description}


\item[[dependencies]]
Lists the dependencies of the UMC. This allows the client
to download all UMC required to use this UMC. The fields mean:
\begin{description}
\item[ID]
	The ID of the UMC to depend upon.

\item[VERSION\_MASK]
	This mask is used to validate a version used. The format is:
	\begin{description}
	\item[$=$VERSION]
		The version number of the dependency exactly match the VERSION.

	\item[$>=$VERSION]
		The version number of the dependency must be greater or equal to
		VERSION.

	\item[$>$VERSION]
		The version number of the dependency must be greater than VERSION.

	\item[$<$VERSION]
		The version number of the dependency must be less than VERSION
	\item[$<=$VERSION]
		The version number of the dependency must be less than or equal to
		VERSION

	\end{description}
	Several values can be used in a mask. They are separated by a space. The
	masks are and'ed and the user is responsible for setting a sane value;
	e.g. `$<$1.0.0 $>$1.0.0' can't match a version.

	If this field is omitted, all versions are allowed.

\end{description}

\item [[state]] These fields are auto-generated by the server. It's the current state of the UMC.

\begin{description}
\item[UPLOADER\_IP\_ADDRESS]
  The ip address of the last uploader of the umc.

\item[TIMESTAMP]
  The timestamp of the last upload. 

\item[DOWNLOADS]
  The number of times the UMC is downloaded.

\item[UPLOADS]
  The number of time a new version of the UMC is uploaded.

\end{description}
\end{description}


\subsection{umc-\emph{lang}.cfg}
\label{section:communication_protocol:server:umc-lang.cfg}.

The configuration file for the translation of the UMC. The
\emph{lang} is replace with the locale of the translation.

\begin{lstlisting}
	[language]
    id = LANGUAGE
    [translated_umc_strings]
      name = NAME
      description = DESCRIPTION
    [/translated_umc_strings]
    [po_file_details]
      size = SIZE
      timestamp = TIMESTAMP
      translated = TRANSLATED
      fuzzy = FUZZY
      untranslated = UNTRANSLATED
    [/po_file_details]
	[/language]
\end{lstlisting}

\begin{description}
\item[ID]
  The id of the language the translation for.

\item[[translated\_umc\_strings]]

  The translation of the name and description, in the language specified by ``ID''.

\begin{description}

\item[NAME]
  See \cref{section:communication_protocol:server:umc.cfg}

\item[DESCRIPTION]
  See \cref{section:communication_protocol:server:umc.cfg}

\end{description}

\item [[po\_file\_details]]
\begin{description}
\item[SIZE]
  The size of the po-file.

\item[TIMESTAMP]
  The timestamp the po-file was last updated.

\item[TRANSLATED]
  The number of translated strings. Must be \textgreater 0.

\item[FUZZY]
  The number of strings that are fuzzy.
  
\item[UNTRANSLATED]
  The number of strings that are not translated.

\end{description}
\end{description}


\section{UMC configuration}
\label{section:communication_protocol:umc}

The pbl file contains the information regarding the addon. Its contents are
stored on the system of the addon developer and on the server. The file
contains the following data:

\begin{lstlisting}
[umc_configuration]
  [info]
    id = ID
    name = NAME
    type = TYPE
    description = DESCRIPTION
    version = VERSION
    authors = AUTHORS
    email = EMAIL
    password = PASSWORD
  [/info]
  [language]
    native_language = NATIVE_LANGUAGE
    translatable = TRANSLATABLE
  [/language]
  [dependencies]?
    [dependency]+
      id = ID
      version? = VERSION_MASK
    [/dependency]
  [/dependencies]
  [image]
    icon_name = ICON_NAME
  [/image]
[/umc_configuration]
\end{lstlisting}

The keys have the following meaning:
\begin{description}
\item[[info]]
	See \cref{section:communication_protocol:server:umc.cfg}.

\item[[language]]
  See \cref{section:communication_protocol:server:umc.cfg}.

\item[[dependencies]]
	See \cref{section:communication_protocol:server:umc.cfg}.

\item[[image]]
\begin{description}
 \item ICON\_NAME The name of the icon that will be displayed in the umc list.
\end{description}

\end{description}

\section{Wire}
\label{section:communication_protocol:wire}

\subsection{Design rational}

\begin{itemize}
 \item The ID is a technical ID and not the couple campaign name + version which is unique too. So we can change the campaign name without breaking dependencies.

\end{itemize}


\subsection{Special packets}
\label{wire:specialpackets}

Special packets are sent in reply and can appears instead an expected packet or in addition of this packet.

\subsubsection{Error packet}
\label{wire:specialpackets:errorpacket}

 The server could not understand the request, or a value in a field of the request is wrong or missing.
In this case an error packet will be sent and it will always take the following form:
\begin{lstlisting}
[error_packet]
  msg = MSG
[/error_packet]
\end{lstlisting}

\begin{description}
 \item MSG Contains a message describing the error.
\end{description}

The common values of MSG are:

\begin{enumerate}
 \item The value $<$value$>$ of the field $<$field\_name$>$ is wrong \{because $<$reason$>$\}.
 \item The value $<$value$>$ of the field $<$field\_name$>$ is missing.
 \item The field $<$field\_name$>$ is unknown.
\end{enumerate}

\subsubsection{Warning packet}
\label{wire:specialpackets:warningpacket}

The server correctly understand the request but the request is not exactly correct for the reasons describe into \textit{msg}.
\begin{lstlisting}
[warning_packet]
  msg = MSG
[/warning_packet]
\end{lstlisting}

\begin{description}
 \item MSG Contains a message describing the warning.
\end{description}

\subsection{Request campaign list}
\label{wire:request_campaign_list}

\subsubsection{Request}
\begin{lstlisting}
	[request_umc_list]
    favlang = LANGUAGE
	[/request_umc_list]
\end{lstlisting}

The field \textit{favlang} is the favorite language in which the add-ons (title and description for example) must be send. 
If the content is not fully (or not at all) translated,  the UMC description is downloaded in its native language, 
and the rest is left to the engine \textit{gettext}.
\newline
\subsubsection{Reply}

An error packet can be sent instead of the umc list packet. No special value appears in this packet. See \cref{wire:specialpackets:errorpacket}.
If the list is empty, no error packet is sent, the list will just appears with no [umc] tag.

\begin{lstlisting}
	[umc_list]
		[umc]*
      [info]
        id = ID
        type = TYPE
        version = VERSION
        size = SIZE
        authors = AUTHORS
      [/info]
      [icon]
        size = SIZE
        extension = EXTENSION
      [/icon]
      [state]
        timestamp = TIMESTAMP
        downloads = DOWNLOADS
        uploads = UPLOADS
      [/state]
      [dependencies]?
        [dependency]+
          id = ID
          version? = VERSION_MASK
        [/dependency]
      [/dependencies]
      [languages]
        [language] +
          id = LANGUAGE
          [translated_umc_strings]
            name = NAME
            description = DESCRIPTION
          [/translated_umc_strings]
          [po_file_details]
            size = SIZE
            timestamp = TIMESTAMP
            translated = TRANSLATED
            fuzzy = FUZZY
            untranslated = UNTRANSLATED
          [/po_file_details]
        [/language]
      [/languages]
		[/umc]
	[/umc_list]
\end{lstlisting}

The fields in [umc] have the following meaning:
\begin{description}
\item[[info]]
  See \cref{section:communication_protocol:server:umc.cfg}

\begin{description}
 \item SIZE The size of the UMC without translation files (.po).
\end{description}

\item[[icon]]

\begin{description}
 \item SIZE The size of the icon.
 \item EXTENSION The extension of the file, or in other word, the file format.
\end{description}


\item[[state]]
  See \cref{section:communication_protocol:server:umc.cfg}.

\item[[dependencies]] 
  Lists the dependencies of the UMC, See \cref{section:communication_protocol:server:umc.cfg}.

\item[[languages]]
  Contains the description of the add-on in the favorite language requested.

If the translation into the favorite language is not available or incomplete, American English is sent.
If American English is not available or complete, the language depends on the native language of this UMC. 
We let the engine \textit{gettext} merge the different file considering this order of preference: $favorite >
American English > native$. So, maximum 3 languages are sent.

\begin{description}
\item [[language]] See \cref{section:communication_protocol:server:umc-lang.cfg}.
\end{description}
\end{description}
The icons are sent in the order of appearance in the [umc\_list] packet, the size of the icon is specified in the [icon] tag.

\subsection{Request umc download}
\label{wire:request_umc_download}

This requests to download an UMC.

\subsubsection{Request}
\begin{lstlisting}
	[request_umc_download]
		id = ID
		language = LANGUAGE
	[/request_umc_download]
\end{lstlisting}

\begin{description}
\item[ID]
	The id of the UMC to download.

\item[LANGUAGE]
	The id of the requested language to download.

\end{description}

\subsubsection{Reply}

An error packet can be sent instead for the common reasons. See \cref{wire:specialpackets:errorpacket}.

\begin{lstlisting}
  [warning_packet]?
    msg = MSG
  [/warning_packet]
	[umc_download]
		id = ID
    size = SIZE
		[languages]
			[language]+
				id = LANGUAGE
        size = SIZE
			[/language]
		[/languages]
	[/umc_download]
\end{lstlisting}

\begin{description}
 \item [[warning\_packet]]
    The warning packet can be sent with the following possible values:
    \begin{enumerate}
      \item The language requested is incomplete.
      \item The language requested is not available.
    \end{enumerate}
    See also \cref{wire:specialpackets:warningpacket}.
  \item[[umc\_download]]
  \begin{description}
    \item [ID] The id of the UMC.
    \item [SIZE] The size of the UMC without translation files (.po).
    \item [[language]]
    \begin{description}
      \item ID The id of the language.
      \item SIZE The size of the translation file (.po).
    \end{description}
  \end{description}
\end{description}


The files send are in the order of appearance in the reply, so first the ID.gz and then
the ID-LANGUAGE.gz.
\begin{description}
\item[ID.gz]
	The gzipped campaign files. The file is packed in a single config file
	to be extracted to the local file-system.

\item[ID-LANGUAGE.gz]
	A gzipped mo-file for a language.

\end{description}


\subsection{Request UMC change password}
\label{wire:request_umc_change_password}

This request changes the password.

\subsubsection{Request}
\begin{lstlisting}
  [request_change_password]
    id = ID
    current_password = CURRENT_PASSWORD
    new_password = NEW_PASSWORD
  [/request_change_password]
\end{lstlisting}

\begin{description}
\item[ID]
  The id of the UMC to change the password.

\item[CURRENT\_PASSWORD]
  The current password of the UMC.

\item[NEW\_PASSWORD]
  The new password of the UMC.
\end{description}

\subsubsection{Reply}
\begin{lstlisting}
  [umc_change_password]
    id = ID
    [error_packet] ?
      msg = MSG
    [/error_packet]
    [warning_packet] ?
      msg = MSG
    [/warning_packet]
  [/umc_change_password]
\end{lstlisting}

\begin{description}
\item[id]
  The id of the UMC to change the password.

\item[[error\_packet]]
  If this packet appears, then the password could not been changed for the reasons described into this packet.
  The possible special reasons are:
  \begin{enumerate}
    \item The current password is wrong.
    \item The time between two requests is not respected.
    \item The new password is too weak.
  \end{enumerate}
  See also \cref{wire:specialpackets:errorpacket} for a description.

\item[[warning\_packet]]
  If this packet appears, the password has been changed but a warning is sent to the user. The warning can take these values:
  \begin{enumerate}
    \item The password is weak, you should use a stronger password.
  \end{enumerate}
  See also \cref{wire:specialpackets:warningpacket}.
\end{description}
If only the ID appears in the packet, the password has been successfully changed.

\subsection{Request a new password}
\label{wire:request_new_password}

\subsubsection{Request}
Request a new password because the user forgot it. A new auto-generated password is sent by email.
\begin{lstlisting}
  [umc_forgot_password]
    id = ID
  [/umc_forgot_password]
\end{lstlisting}
\begin{description}
 \item [ID] The ID of the UMC that the password has been forgotten.
\end{description}

\subsubsection{Reply}

The reply can be an error packet for the common reasons. See \cref{wire:specialpackets:errorpacket}.

\begin{lstlisting}
  [umc_forgot_password]
    id = ID
    mail = MAIL
  [/umc_forgot_password]
\end{lstlisting}
\begin{description}
  \item [ID] The ID of the UMC of the forgotten password.
  \item [MAIL] The address mail where the password has been sent.
\end{description}

\subsection{Request umc upload}
\label{wire:request_umc_upload}

Request an upload of a new or an updated version of a UMC.
A new version have no ID in the .pbl file while an updated version have an ID.

\subsubsection{Request}
\begin{lstlisting}
[request_umc_upload] 
  size = SIZE
  format = FORMAT
  [umc_configuration]
    [info]
      id = ID
      name = NAME
      type = TYPE
      description = DESCRIPTION
      version = VERSION
      authors = AUTHORS
      email = EMAIL
      password = PASSWORD
    [/info]
    [language]
      native_language = NATIVE_LANGUAGE
      translatable = TRANSLATABLE
    [/language]
    [dependencies]?
      [dependency]+
        id = ID
        version? = VERSION_MASK
      [/dependency]
    [/dependencies]
    [image]
      icon_name = ICON_NAME
    [/image]
  [/umc_configuration]
[/request_umc_upload]
\end{lstlisting}
\begin{description}
  \item [SIZE] The size the complete UMC that is sent next.
  \item [FORMAT] The format of the UMC, this field can take these values: gz or bz2.
  \item [UMC\_CONFIGURATION] This part is commonly known as the ``pbl file''. See \cref{section:communication_protocol:umc} for details.
\end{description}

The server will decompressed or unpack this file and will look recursively into the repository to check
if all the required files are present in respect of the pattern described in their respective section.

The pack must contains, at least:

\begin{enumerate}
  \item A .pbl file.
\end{enumerate}

But can also contains:

\begin{enumerate}
 \item An icon file, the name of the icon file is specified in the .pbl file, so the server will look recursively into the archive.
\end{enumerate}


\subsubsection{Reply}

An error packet can be sent for the common reasons, see \cref{wire:specialpackets:errorpacket}.
But also for specific reasons:

\begin{enumerate}
  \item The translatable flag in the .pbl has been set, but no American English translation can be found.
  \item The file $<$file\_type$>$ is malformed: $<$reason$>$.
  \item A conflict has been detected: $<$reason$>$.
  \item Wrong password.
\end{enumerate}

If no error packet appears, in case of a new version, the ID field in the .pbl is add and
the following response is sent:

\begin{lstlisting}
[request_umc_upload]
  [umc]
    [info]
      id = ID
      type = TYPE
      version = VERSION
      size = SIZE
      authors = AUTHORS
    [/info]
    [icon]
      size = SIZE
      extension = EXTENSION
    [/icon]
    [state]
      timestamp = TIMESTAMP
      downloads = DOWNLOADS
      uploads = UPLOADS
    [/state]
    [dependencies]?
      [dependency]+
        id = ID
        version? = VERSION_MASK
      [/dependency]
    [/dependencies]
    [languages]
      [language] +
        id = LANGUAGE
        [translated_umc_strings]
          name = NAME
          description = DESCRIPTION
        [/translated_umc_strings]
        [po_file_details]
          size = SIZE
          timestamp = TIMESTAMP
          translated = TRANSLATED
          fuzzy = FUZZY
          untranslated = UNTRANSLATED
        [/po_file_details]
      [/language]
    [/languages]
  [/umc]
[/request_umc_upload]
\end{lstlisting}

\begin{description}
  \item [[umc]] In case of a new version, the .pbl file is updated with the ID. See \cref{wire:request_campaign_list}.
\end{description}

\subsection{Request umc delete}
\label{wire:request_umc_delete}

\subsubsection{Request}
\begin{lstlisting}
[request_umc_delete]
  id = ID
  password = PASSWORD
[/request_umc_delete]
\end{lstlisting}
\begin{description}
  \item [ID] The ID of the UMC we want to delete.
  \item [PASSWORD] The password of the UMC.
\end{description}

\subsubsection{Reply}

An error packet can be sent for the common reasons (see \cref{wire:specialpackets:errorpacket}) but also because:
\begin{enumerate}
 \item The password is wrong.
\end{enumerate}

Otherwise, a packet with no field is sent.
\begin{lstlisting}
[request_umc_delete]
[/request_umc_delete]
\end{lstlisting}

\subsection{Request license}
\label{wire:request_license}

\subsubsection{Request}
\begin{lstlisting}
[request_license]
  language = LANGUAGE
[/request_license]
\end{lstlisting}

\begin{description}
 \item [LANGUAGE] The language we want the license. 
\end{description}

\subsubsection{Reply}

The license is sent in the language asked or, if not available, in American English.

\begin{lstlisting}
[request_license]
  [warning_packet] ?
    msg = MSG
  [/warning_packet]
  [translated_text] ?
    [disclaimer]
      ENGLISH_DISCLAIMER
    [/disclaimer]
    [translated_license]
      TRANSLATED_LICENSE
    [/translated_license]
  [/translated_text]
  [english_text]
    ENGLISH_LICENSE
  [/english_text]
[/request_license]
\end{lstlisting}
\vspace{15pt}

\textbf{Note :} The text is not a key value as it can be on multiple lines. We enclosed text in specific markup.
\begin{description}
 \item [[warning\_packet]] Can be sent if the language of the license is not the language expected.
 \item[[translated\_text]] is sent only if there is no warning message before.
 \item [ENGLISH\_DISCLAIMER] A english disclaimer stating that the translated license has no judicial value.
 \item [TRANSLATED\_LICENSE] The translated license in the language requested.
 \item [TEXT] The text of the license in english.
\end{description}

\section{Text encoding}

The text encoding is important because the server can open the file to verify their validity or to read
information in the .pbl file.

The strings could be in something else than American English and for this reason, the encoding must be UTF-8.

\section{Security}
\label{section:security}

\subsection{Password storage}
\label{section:security:passwordstorage}

The passwords will be stored with a strong hash function such as SHA-512, a salt only knowable by the server is append to this hash. 
\newline

For example, the library OpenSSL (http://www.openssl.org/docs/crypto/sha.html) provides the algorithm, but an encapsulation in a C++ class would be required.

\subsection{Password transmission}
\label{section:security:passwordtransmission}

The password will be transmit via Transport Layer Security (TLS) protocol.
\newline

For example, the Boost.Asio library provides tools to transmits data over a ssl layer.